% ----------------------------------------------------------------
% Report Class (This is a LaTeX2e document)  *********************
% ----------------------------------------------------------------
\documentclass[11pt]{report}
\usepackage[english]{babel}
\usepackage{amsmath,amsthm}
\usepackage{amsfonts}
\usepackage{amssymb}%
\usepackage{graphicx}%
\usepackage{tabularx}
\usepackage{epstopdf}

\addtolength{\textheight}{3.2cm}
\addtolength{\textwidth}{1.5cm}
\addtolength{\marginparwidth}{-2.2cm}
\addtolength{\oddsidemargin}{-1.2cm}
\addtolength{\headheight}{-2.4cm}
%\addtolength{\footskip}{-6cm}

% THEOREMS -------------------------------------------------------
\newtheorem{thm}{Theorem}[chapter]
\newtheorem{cor}[thm]{Corollary}
\newtheorem{lem}[thm]{Lemma}
\newtheorem{prop}[thm]{Proposition}
\theoremstyle{definition}
\newtheorem{defn}[thm]{Definition}
\theoremstyle{remark}
\newtheorem{rem}[thm]{Remark}
% ----------------------------------------------------------------
\begin{document}

%\title{STATS 782, 2017}
\setlength\extrarowheight{3pt}
\begin{tabular*}{\textwidth}{ @{} l @{\extracolsep\fill} c @{\extracolsep\fill} r @{}}
  \hline
  % after \\: \hline or \cline{col1-col2} \cline{col3-col4} ...
  \textbf{STATS 782, 2017} & \textbf{Assignment1} & \textbf{Date: 2017-08-15} \\
   & Zhi Zhang, 708439475, zzha822 & \\
  \hline
\end{tabular*}
%\author{Zhi Zhang, $\ $708439475, $\ $zzha822
%\\The University of Auckland}
%\date{August 3, 2017}
%\maketitle

\begin{enumerate}
    \item[1.] The answers are below:
    \begin{verbatim}> fib = function(n) {
+   s = numeric(n)
+   
+   if (n <= 1) s[n] = 0 
+   else {
+     s[1:(n - 1)] = fib(n - 1)
+     if (n == 2) s[n] = 1
+     else s[n] = s[n - 1] + s[n - 2]
+   }
+   
+   s
+ }
> 
> fib(1)
[1] 0
> fib(2)
[1] 0 1
> fib(3)
[1] 0 1 1
> fib(10)
 [1]  0  1  1  2  3  5  8 13 21 34\end{verbatim} 
 
    \item[2.] The answers are below:
    \begin{enumerate}
    	\item[(a)] \begin{verbatim}> clusters.medians = function(x, c) {
+   lenc = length(c)
+   
+   d = outer(c, x, function(cj, xi) abs(xi - cj))
+   d.minnum = apply(d, 2, which.min)
+   
+   con = outer(1:lenc, d.minnum, function(num, minnum) num == minnum)
+   
+   xv = unlist(apply(con, 1, function(t) median(x[t])))
+   xv
+ }
> 
> find.clusters.medians = function(x, c) {
+   ctmp1 = c
+   repeat {
+     ctmp2 = clusters.medians(x, ctmp1)
+     if (all(abs(ctmp1 - ctmp2) < 1e-07)) break
+     else ctmp1 = ctmp2
+   }
+   ctmp1
+ }
> 
> x = faithful$eruptions
> find.clusters.medians(x, c(2,4))
[1] 1.9830 4.3415\end{verbatim}
 	\item[(b)] \begin{verbatim}> find.clusters.medians(x, c(2,3,4))
[1] 1.9830 3.9665 4.5330\end{verbatim}
 	\item[(c)] \begin{verbatim}> find.clusters.medians(x, c(2,3,4,5))
[1] 1.967 3.600 4.150 4.600\end{verbatim}
    \end{enumerate}
    
    \item[3.] The answers are below:
    \begin{verbatim}> sign.matrix = function(x) outer(x, x, function(x1, x2) sign(x1 - x2))
> 
> conc = function(x, y) {
+   conc.mtx = sign.matrix(x)
+   conc.mty = sign.matrix(y)
+   conc.z = conc.mtx + conc.mty
+   c = length(which(conc.z < 0 | conc.z > 0))
+   n = length(x)
+   c / (n * (n - 1)) 
+ }
> 
> conc(x = 1:5, y = c(3, 1, 4, 5, 2))
[1] 0.6
> 
> set.seed(782)
> x = round(rnorm(1000))
> y = x + round(rnorm(1000))
> conc(x, y)
[1] 0.8518939\end{verbatim}
 	
    \item[4.] The answers are below:
    \begin{enumerate}
    	\item[(a)] \begin{verbatim}\end{verbatim}
 	\item[(b)] \begin{verbatim}\end{verbatim}
 	\item[(c)] \begin{verbatim}\end{verbatim}
    \end{enumerate}
    
\end{enumerate}

\end{document}
% ----------------------------------------------------------------
